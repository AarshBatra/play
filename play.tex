\documentclass{article}

\title{Hierarchal structures in LaTeX \thanks{to Aarsh Batra}}
\author{Aarsh Batra}
\date{April, 26, 2018}



\begin{document}
\pagenumbering{gobble}
\maketitle
\tableofcontents

\newpage
\pagenumbering{arabic}
\section{Introduction to Mathematics}
Mathematics is the language in which Nature speaks. In order to understand nature, one has to understand its language. So far we have figured out a lot of what nature is conveying to us. But we have just understood so liitle in comparison to what nature has to offer. This quest is our best hope towards understanding who we really are. Be Curious.

\section{Branches of Mathematics}
So far we have divided Mathematics into various subfields which can be thought of analogous to our alphabets that make up our languages. Some of the fields include: Geometry, Algebra, Calculus, Topology. Let's take a look at each one of them.

\subsection{Geometry}
\paragraph{Geometry is the study of visual (both abstract and literally visual) structures in nature. It could be a 2D shape like a square or a 3D surface of a cube or any n dimensional structure.}
\subparagraph{2D}
\subparagraph{3D}
\subparagraph{ND}

\subsection{Algebra}

\subsection{Calculus}

\subsubsection{Differential Calculus}

\subsubsection{Integral Calculus}

$$\lambda \sin{\theta} {{e^i}^j}_{l\dots k}$$

$ e^{i\pi} = -1 $ or in other words $e^{i\pi} + 1 = 0$


\end{document}